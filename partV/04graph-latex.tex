
% latex nombre_archivo.tex
% dvipdf nombre_archivo.dvi
\documentclass{article}
\usepackage{pstricks}
\usepackage{xcolor}

\title{My document about graphics}
\author{J. A. Orduz-Ducuara}
\begin{document}
\maketitle
\section{My first document on graphics with \LaTeX}
In this document we will explore 
the \verb|pstricks| package.
We study some options, the reader has to check the information 
and more
(for instance in my mac: 
/usr/local/texlive/2017/texmf-dist/doc/generic/pstricks
)
The general form to implement the 
tool is:\\

\begin{verbatim}
\begin{pspicture}([Xmin],[Ymin])([Xmax],[Ymax])
...[COMANDOS]
\end{pspicture}
\end{verbatim}

\subsection{How can I create a vertical line?}
\begin{pspicture}(6,3)
  \psline[linestyle=dashed](6, 0)(6, 3)
 \end{pspicture}
 
\noindent where we use
with \verb|\begin{pspicture}(x0, y0)(x1, y1)|
is the size of the picture. 
And to create a line
\begin{verbatim}
\begin{pspicture}(3,3)
  \psline[linestyle=dashed](0, 0)(0, 3)
 \end{pspicture}
\end{verbatim}

We can change the options and introduce other:



or we can explore other line type:
\begin{pspicture}(4,3) \psset{arrowscale=2,linewidth=1pt}
\psline{]-[}(4,0)
\psline{)-(}(0,1)(4,1)
\psline{)->}(0,2)(4,2)
\psline{]->>}(0,3)(4,3)
\end{pspicture}





We should plot a complex graphics.\\
\begin{pspicture}(3,3)
  \psline{<->}(0, 0)(3, 0)
 \end{pspicture}

\begin{verbatim}
\begin{pspicture}(3,3)
  \psline{<->}(0, 0)(3, 0)
 \end{pspicture}
\end{verbatim}


or \\
\begin{pspicture}(3,3)
  \psline{<->}(0, 0)(3, 0)
\end{pspicture}

\begin{verbatim}
\begin{pspicture}(3,3)
  \psline{<->}(0, 0)(3, 0)
\end{pspicture}
\end{verbatim}

and\\

\begin{pspicture}(3,3)
  \psline{<->}(0,3)(0, 0)(3, 0)
 \end{pspicture}


\begin{verbatim}
\begin{pspicture}(3,3)
  \psline{<->}(0,3)(0, 0)(3, 0)
 \end{pspicture}
\end{verbatim}

\subsection{To create circles}
  

\begin{pspicture}(0.5\linewidth,4)
  \pscircle[fillstyle=vlines,hatchangle=0,hatchsep=0.6pt,%
    hatchwidth=1pt,hatchwidthinc=0.3pt,hatchangle=90,
    hatchcolor=red](2,2){2}
  \pscircle[fillstyle=vlines,hatchangle=0,hatchsep=0.6pt,%
    hatchwidth=1pt,hatchwidthinc=0.3pt,hatchangle=-45,
    hatchcolor=green](7,2){2}
  \pscircle[fillstyle=hlines,hatchangle=0,hatchsep=0.6pt,%
    hatchwidth=1pt,hatchwidthinc=0.3pt,hatchangle=45,
                                                 hatchcolor=blue](12,2){2}
\end{pspicture}


\begin{pspicture}(-5, -5)(5, 5)
\pscircle[linecolor=blue,linestyle=dashed](1,1.5){1}
\pscircle[linecolor=blue,linestyle=dashed](2,1.5){1}
\pscircle[linecolor=blue,linestyle=dashed](3,1.5){1}
\pscircle[linecolor=blue,linestyle=dashed](4,1.5){1}
\pscircle[linecolor=blue,linestyle=dashed](5,1.5){1}
\end{pspicture}

where we use
\begin{verbatim}
\begin{pspicture}(0, 0)(0, 5)
\pscircle[linecolor=blue,linestyle=dashed](1,1.5){1}
\pscircle[linecolor=blue,linestyle=dashed](2,1.5){1}
\pscircle[linecolor=blue,linestyle=dashed](3,1.5){1}
\pscircle[linecolor=blue,linestyle=dashed](4,1.5){1}
\pscircle[linecolor=blue,linestyle=dashed](5,1.5){1}
\end{pspicture}
\end{verbatim}

\end{document}
